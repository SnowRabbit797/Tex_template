\documentclass{jsarticle}

%%
%% パッケージを追加する際はここに記述
%%
\usepackage{latexsym}
\usepackage{mathrsfs}
\usepackage{mathtools}
\usepackage{listings}
\usepackage{url}
\usepackage{ascmac}



%%%%%%%%%%%%%%%%%%%%%%%%%%%%%%%%%%%%%%%%%%%%%% 
%% ここから 触らない
%%%%%%%%%%%%%%%%%%%%%%%%%%%%%%%%%%%%%%%%%%%%%% 

\makeatletter
% 文章の種類を記述 
\def\@thesis{後期発表資料}
\def\id#1{\def\@id{#1}}
\def\mail#1{\def\@mail{#1}}
\def\department#1{\def\@department{#1}}

\def\@maketitle{
\begin{center}
{\huge \@thesis \par}
\vspace{40mm}
{\LARGE\bf \@title \par}
\vspace{10mm}
{\Large \@date\par}
\vspace{20mm}
{\Large \@department \par}
{\Large 学籍番号 \@id \par}
{\Large \@mail \par}
\vspace{10mm}
{\large \@author}
\end{center}
\par\vskip 1.5em
}

\makeatother

%%%%%%%%%%%%%%%%%%%%%%%%%%%%%%%%%%%%%%%%%%%%%% 
%% ここまで触らない
%%%%%%%%%%%%%%%%%%%%%%%%%%%%%%%%%%%%%%%%%%%%%% 

% ここにレポートタイトルを記述
\title{数理最適化とWEBアプリケーション} 
% 提出日を記述
\date{2023年9月28日} 
\department{日本大学理工学部 数学科 4年}
% 学籍番号を記述
\id{0050} 
% メールアドレスを記述
\mail{csao20050@g.nihon-u.ac.jp} 
% 名前を記述
\author{佐藤 葵}

\begin{document}
\maketitle

\newpage

%%%%%%%%%%%%%%%%%%%%%%%%%%%%%%%%%%%
%% 以下 自由に改変
%%%%%%%%%%%%%%%%%%%%%%%%%%%%%%%%%%%
\tableofcontents
\clearpage

\section{前期の内容}
\vspace\baselineskip	
\subsection{前期の取り組み}
数理最適化とは、数学的な手法やアルゴリズムを用いて、与えられた制約条件の下で特定の目的関数を最大化または最小化する問題を解決に導く計算技術のこと。前期はオーム社から出版されている「Pythonではじめる数理最適化」\cite{参考書}を参考にして、実社会で起きている身近な問題の制約・目的を考え、Pythonのライブラリである"Pulp"や"Pandas"を利用して計算する手法を学んだ。

\subsection{数理モデルとは}
{\bf 数理モデル}とは、{\bf「対象を数学によって記述したモデル」}のこと。 現実にある特定の動きや現象を数式化できれば、現実で試行することなくコンピューター上でシミュレーションをすることができ、技術が進歩するほどに"数理モデル"は必要とされる。
\par 例えば、新型コロナウイルス感染症が流行した際には、数理モデルを使った感染予測が各所から発表され、対策をたてる際に有効活用された。Googleが発表している{\bf「新型コロナウイルス感染症(COVID-19)の感染予測(日本版)」}では、数理モデルとAIを組み合わせこれから28日の感染予測をグラフ表示させることができる。

\subsection{数理最適化とは}
{\bf 数理最適化}とは、記述した数理モデルから、制約条件\footnote{守らなければいけない条件}を満たしつつ、コストの最小化や利益が最大化されるような変数の値を求める手法のこと。数理最適化を現実の問題に適用するには、まず最適化したい現実の問題に関して「目的関数\footnote{最適化問題の中で、最大化あるいは最小化したい関数のこと}」と「制約条件」を数理モデルとして定式化する。その後、その数理モデルを解くのに適した最適化アルゴリズムを選択し、目的関数を最大化or最小化する変数の値を算出する。\par
{\bf 線形計画問題}とは、決定変数\footnote{最適化の対象になる変数のこと}が連続変数で、制約条件や目的関数が全て線形の式\footnote{変数の一次式で表したもの}で表現された最適化問題のこと。その線形計画問題を解く方法・アルゴリズムを総称したものを{\bf 線形計画法}という。

\newpage

\subsection{学校のクラス編成問題}
\subsubsection{導入}
前期の内容を少し振り返る。小学校や中学校の頃、年度の変わり目にはクラス替えを実施していた。一口に学校と言っても、公立に私立、クラス数も2クラスだったり7クラスあったり。クラス編成のルールは多岐にわたれど、大抵の学校ではルールが共通していそうなもの。
\par 一般的にはこのようなルールが考えられる。
\begin{quote}
 	\begin{itemize}
 		 \item 各クラスの男女比を同程度にする
 		 \item 学力の偏りがないようにする
		 \item 各クラスに1人はピアノが弾ける子を入れる
		 \item 兄弟は同じクラスにしない
 	\end{itemize}
\end{quote}
\par 他にも考えればいくらでも出てきそうだが、それらの制約を用いて手動で解くのはは容易ではない。このような場合、数理最適化問題をコンピューター上で解くことにより、かなり楽に自動化できる。\\
このようなクラス編成問題を数理最適化問題にモデリングして解くメリットは次の4つが考えられる。

\begin{quote}
 	\begin{itemize}
 		 \item {\bf 最 適 性}:最適なクラス編成ができる
 		 \item {\bf コスト削減}:自動化することにより、教員の作業コストを削減できる
		 \item {\bf 柔 軟 性}:クラス編成のルールを柔軟に変更できる
		 \item {\bf 公 平 性}:感情に左右されず、恣意的な操作をせずにクラス編成できる。
 	\end{itemize}
\end{quote}

\subsubsection{課題整理}
今回は公立中学校のある学年のクラス編成について、具体的な例からそれぞれのルールを考えていく。
\begin{itembox}[l]{{\bf 1.公立中学校のクラス編成}}
	学年の全生徒をそれぞれ一つのクラスに割り当てる
\end{itembox}\\	


実際の生徒数とクラス数を定義する。今回の問題では、1学年に318人の生徒が在籍し、8つのクラスがあるとする。8クラスに318人の生徒を均等に分けると、40人のクラスが6クラス、39人のクラスが1クラス作れる。つまり、生徒数は39人以上40人いかと定めることが出来る。
\begin{itembox}[l]{{\bf 2.学年には318人の生徒がいて、8つのクラスがある}}
	各クラスの生徒の人数は39人以上40人以下とする
\end{itembox}\\


ただ、2のルールだけでは女子だけのクラス、男子だけのクラスを作ることもできてしまうので次のルールを追加する。
\begin{itembox}[l]{{\bf 3.学年には男子生徒が158人、女子生徒が160人いる}}
	各クラスの男子生徒、女子生徒の人数は20人以下とする
\end{itembox}\\


次に学力に関するルールを設定する。生徒が300人いればそれだけ学力に開きが出てしまう。今回は全クラスを普通クラスと設定するので、出来るだけ各クラスの学力の偏りは少なくしたい。すると次のようにルール化できる。
\begin{itembox}[l]{{\bf 4.学力試験は500点満点で、平均点は303.6点}}
	各クラスの学力試験の平均点は学年平均 ±10点とする
\end{itembox}\\


クラス運営において、リーダー気質の生徒がいることがとても重要で、年間通してあらゆる行事でもクラスをまとめてくれる人がいるととても役に立ちます。他にもピアノを弾ける子だったりと、ある特定の才能がある子達はクラス編成において分散させたいところ。今回は"リーダー気質の生徒"に着目し、以下のルールを設定する。
\begin{itembox}[l]{{\bf 5.学年にはリーダー気質の子が17にいる}}
	各クラスにリーダー気質の生徒を2人以上割り当てる。
\end{itembox}\\

次に少しデリケートな問題を考える。不登校・体の不自由な人・欠席が多い人など、特定の支援が必要であったり学校特有の事情を持つ生徒に関しては出来るだけばらけさせたい。
\begin{itembox}[l]{{\bf 6.学年に特別な支援が必要な生徒が4人いる}}
	特別な支援が必要な生徒は各クラスに1人以下とする
\end{itembox}\\


兄弟や同姓同名の子が同学年にいる場合、そのようなある特定のペアを同じクラスに割り当てないようにしたい。最後にこのようなルールを追加する。
\begin{itembox}[l]{{\bf 7.学年に特定ペアが3組いる}}
	特定のペアの生徒は同一クラスに割り当てない
\end{itembox}\\

このように今回は7つの課題から要件を設定して、数理モデリングしていくことにする。

%%%%%%%%%%%%%%%%%%%%%%%%%%%%%%%%%%%%%%%%%
%5page-6page
%%%%%%%%%%%%%%%%%%%%%%%%%%%%%%%%%%%%%%%%%


%%%%%%%%%%%%%%%%

%ハードマージン・ソフトマージン
%%%%%%%%%%%%%%%

\newpage

\subsubsection{数理モデリングと実装}
1.4.2で整理した課題を元に数理モデルの構築を行う。

\begin{itembox}[l]{{\bf 1.学年の全生徒をそれぞれ一つのクラスに割り当てる}}
	\begin{itemize}
		\item {\bf 生徒のリスト}:$S$
		\item {\bf クラスのリスト}:$C$
		\item {\bf 変数}:$x_{s,c} \in\{0,1\} (s \in S, c \in C)$
		\item {\bf 制約式(1)}:${\displaystyle \sum_{c \in C}x_{s,c}=1} (s \in S)$
	\end{itemize}
\end{itembox}
\par まず生徒のリスト$S$とクラスのリスト$C$を定義する。生徒のリスト$S$は1〜318の間の値をとる学生番号で、クラスのリスト$C$は8クラスを表している。今回、クラスは1〜8ではなくA〜Hとする。
次に、数理最適化問題で最も重要な決定変数$x$を定義する。{\bf 決定変数}とは最適化の対象となる変数のことで、目的を達成するために操作する変数のこと。
\par 今回の問題では各生徒をどのクラスに割り当てるかを考えるので$s \in S$と$c \in C$の組み合わせに対して$x_{s,c}$を定義し、「$x_{s,c}=1$の場合は生徒sをクラスcに割り当てる」「$x_{s,c}=0$の場合は生徒sをクラスcに割り当てない」ことを表すことにする。例えば、7番の生徒をEクラスに割り当てる場合、$x_{7,E}=1$となる。ただ、Cクラスを軸に見てみると、7番の生徒はCクラスではないので$x_{7,C}=0$となる。
\par 当たり前のことだが、それぞれの生徒は1つのクラスにしか割り当てることができない。先ほどの例だと134番の生徒を固定した場合、$x_{7,A}$、$x_{7,B}$、$x_{7,C}$、$x_{7,D}$、$x_{7,E}$、$x_{7,F}$、$x_{7,G}$、$x_{7,H}$の8個の変数のうち、1となるのは1つだけ。
\par 数式で表現すると
\vspace\baselineskip	
\par $x_{7,A}+x_{7,B}+x_{7,C}+x_{7,D}+x_{7,E}+x_{7,F}+x_{7,G}+x_{7,H} = 1$
\vspace\baselineskip	
\par となり、${\displaystyle \sum_{c \in C}x_{s,c}=1}$が成立する。これを制約式として追加する。

\par 次は(2)のモデリング。
\begin{itembox}[l]{{\bf 2.各クラスの生徒の人数は39人以上40人以下とする}}
		${\qquad \displaystyle \sum_{s \in S}x_{s,c} \geq  39} (c \in C)$\\
		${\qquad \displaystyle \sum_{s \in S}x_{s,c} \le 40} (c \in C)$
\end{itembox}
\par 例としてクラスcをAで固定した時、$x_{1,A}、 x_{2,A}、 …、 x_{318,A}$を利用して要件を定義する。つまり上の式は\\
$x_{1,A}+ x_{2,A}+ …+x_{318,A}\geq 39$、すなわち、${\displaystyle \sum_{s \in S}x_{s,c} \geq  39}$と表現できる。実際にはクラスをAに固定するわけではないので、$(c \in C)$でクラスcについての制約式で表現した。

\newpage

\par 同じく(3)についてもモデリングする。

\begin{itembox}[l]{{\bf 3.各クラスの男子生徒、女子生徒の人数は20人以下とする}}
		${\qquad \displaystyle \sum_{s \in S_{male}}x_{s,c} \le  20} (c \in C)$\\
		${\qquad \displaystyle \sum_{s \in S_{female}}x_{s,c} \le 20} (c \in C)$
\end{itembox}
$S_{male}、S_{female}$を生徒リストの中から男子生徒・女子生徒を抽出して新たに作られた定義として扱う。これが基本的な形式になっており、この後にも同じ形式で表現できる制約式が出てくる。

\vspace\baselineskip	

次に(4)のモデリングをしていく。
\begin{itembox}[l]{{\bf 4.各クラスの学力試験の平均点は学年平均点 ±10点とする}}
	\begin{itemize}
		\item {\bf 定数(各生徒の学力)}:$score_{s} (s \in S)$
		\item {\bf 定数(学年平均点)}:${score\_mean}$
		\item {\bf 制約式(4):各クラスの学力試験の平均点は学年平均点±10点とする}
		\\ $score\_mean-10 \le \frac{\displaystyle \sum_{s \in S}score_{s}・x_{s,c}}{\displaystyle \sum_{s \in S}x_{s,c}} (c \in C)$
		\\ $\frac{\displaystyle \sum_{s \in S}score_{s}・x_{s,c}}{\displaystyle \sum_{s \in S}x_{s,c}} \le score\_mean+10 (c \in C)$
	\end{itemize}
\end{itembox}

各生徒の学力を$score_{s}$、学年平均点を${score\_mean}$で与えた時、各クラスの平均点は、クラスに割り当てられた生徒の点数の和($\displaystyle \sum_{s \in S}score_{s}・x_{s,c}$)をクラスに割り当てられた人数($\displaystyle \sum_{s \in S}x_{s,c}$)で割ることで以下のように表すことができる。

\begin{equation}
	\frac{\displaystyle \sum_{s \in S}score_{s}・x_{s,c}}{\displaystyle \sum_{s \in S}x_{s,c}}
\end{equation}

全校生徒は318人だが、例として全校生徒を4人として考える。

\begin{itemize}
	\item Cクラスに所属するaさん(学力300点)
	\item Aクラスに所属するbさん(学力320点)
	\item Aクラスに所属するcさん(学力280点)
	\item Bクラスに所属するdさん(学力260点)
\end{itemize}
Aクラスで固定した時、(1)の式で簡単に考えると以下のようなる。\\ \\
\begin{equation}
	\displaystyle \frac{300・0+320・1+280・1+260・0}{2人}=300点 \notag
\end{equation}
\newpage
学力試験は500点満点で平均点は303.6点としており、今回の制約式では平均点の±10以内で定義するので、今回の例ではたった2人ではあるが、一応制約を満たしてはいる。

しかし、$score\_mean-10 \le \frac{\displaystyle \sum_{s \in S}score_{s}・x_{s,c}}{\displaystyle \sum_{s \in S}x_{s,c}} (c \in C)$の式の右辺には分母分子に変数が現れているため非線形の式になっている。非線形の制約が含まれると、利用できる数値計算ソフトウェアが限定されてしまうので、非線形の制約を線形の制約に変換可能な場合は、線形に書き換えるのが定石となっている。
上式を線形の式に変換するには通分すれば良い。\\
\begin{align}
	(score\_mean-10)・ \displaystyle \sum_{s \in S}x_{s,c} \le \displaystyle \sum_{s \in S}score_{s}・x_{s,c} \qquad (c \in C) \\ \notag \\
	\displaystyle \sum_{s \in S}score_{s}・x_{s,c} \le (score\_mean+10)・\displaystyle \sum_{s \in S}x_{s,c}  \qquad (c \in C)
\end{align}
\vspace\baselineskip	

続いて、要件5,6のモデリングをするのだが、形式は要件2,3と変わらないので説明は省く。


\begin{itembox}[l]{{\bf 5.各クラスにリーダー気質の生徒を2人以上割り当てる}}
	\begin{itemize}
		\item リーダー気質の生徒リスト:$S_{leader}$ 
		\item 制約式(5) 各クラスにリーダー気質の生徒を2人以上割り当てる \\
		${\qquad \displaystyle \sum_{s \in S_{leader}}x_{s,c} \geq 2} (c \in C)$
	\end{itemize}
\end{itembox} \\
\begin{itembox}[l]{{\bf 6.特別な支援が必要な生徒は各クラスに1人以下とする}}
	\begin{itemize}
		\item 特別な支援が必要な生徒のリスト:$S_{support}$ 
		\item 制約式(6) 特別な支援が必要な生徒は各クラスに1人以下とする \\
		${\qquad \displaystyle \sum_{s \in S_{support}}x_{s,c} \le 1} (c \in C)$
	\end{itemize}
\end{itembox}\\

最後に要件7をモデリングする。
\begin{itembox}[l]{{\bf 7.特定ペアの生徒は同一クラスに割り当てない}}
	\begin{itemize}
		\item 生徒の特定のペアリスト:$SS$ 
		\item 制約式(7) 特定ペアの生徒は同一クラスに割り当てない \\
		$x_{s1,c}+x_{s2,c} \le 1 \qquad(c \in C, (s1,s2) \in SS)$
	\end{itemize}
\end{itembox} \\

具体的な例を用いて考える。特定のペアは何組かいるのだが、例として生徒番号53と生徒番号211を兄弟として考えてみる。要件で定義したが、$x_{s,c}$は1か0の値のみ取り得る。例によってクラスをAに固定してみると、$(x_{53,A},x_{211,A})$は(0,0),(1,0),(0,1),(1,1)の4通りの組み合わせが考えられる。値が(1,1)の場合、これは両者が同じクラスにいるということ。つまり、$x_{53,A}+x_{211,A}$=2になってしまうので、今回の制約では$x_{s1,c}+x_{s2,c} \le 1 \qquad(c \in C, (s1,s2) \in SS)$となっている。
\par このように定式化してきた7つの要件をPythonで実装する。


\subsubsection{Pythonで実装/まとめ}
今回のクラス編成問題はゼミで使用している本\cite{参考書}を参考にしており、指定のURLからcsv\footnote{文字や記号で構成された、各項目間がカンマ(,)で区切られたデータのこと}形式のファイルをダウンロードすることで、円滑に実装する過程を学習することができた。実装にはPythonライブラリの中から、数理最適化問題を解くためのPulpやPandasを利用した。使用したcsvファイルとコードについてはGoogleDriveにアップロードするので各自でご覧いただきたい。

\newpage

\section{後期の目標}
前期に引き続き、本に記載の実社会で起きている身近な問題をPythonを用いて解決していくとともに、身近に潜む問題について、一から制約・目的を考えてPythonで実装していく。最終的には実装した数理最適化モデルをAPIとして公開し、Webアプリケーションを通してブラウザから操作することで、誰でも機能を利用できるように開発を進めていく。

\vspace\baselineskip	


\section{APIとは}
{\bf API(API連携)}とは、アプリケーション・プログラミング・インターフェース(Application Programing Interface)の頭文字をとったもので、API連携を活用すると、1からプログラミングする必要もなく、機能・サービスの向上が可能になり、高度なプログラミングを要求されることなく、幅広いソフトウェアの開発も可能となる。
他にも、自社の開発したアプリケーションに他社の開発したアプリケーションの機能を埋め込むこともできる。たとえば、自社のサイトに最新の地図を載せたい時、GoogleマップのAPIを利用すると、常に最新の地図を掲載することが可能になる。また、TwitterとAPI連携すれば、自社の製品に対するツイートをリアルタイムで自動的に掲載ができるようになる。このように、API連携を活用するとさまざまなメリットが生じる。

\subsection{APIの使用例}
ここでAPIの使用例として、新規開業するホテルのサイトを作りたい時、サイトにどのような機能を実装して、その機能を実装するためにどのAPIを使えばいいか考えてみる。ここで、このホテルは"品川駅前"の"会員制ホテル"とする。
\begin{quote}
 	\begin{itemize}
 		 \item {\bf 会員登録}・・・今回のホテルに限らず、ECサイトなど会員登録をする際に、GoogleやTwitterのアカウントでログインをすることが多々ある。これは会員登録の機能をプログラミングせずに、Googleなどのアカウント機能とAPI連携しているため。新たにログイン認証システムを構築することなく、既存の大手企業が制作したアカウント機能を利用することで、ユーザーからの信頼度も高くなり、集客に繋がる。
 		 \item {\bf アクセスマップ}・・・あらゆる施設のサイトで使われている、その施設に行くまでのルートをサイト上にマップ表示している例がよくある。ホテルのサイトにも同様に、最寄りの駅やバス停からのルートマップをサイトに出せば、利用客に対してアクセスの方法をわかりやすく提示することができる。地図APIを利用すれば実現することができ、例えばGoogleMapsApiではサイト上にGoogleMapsを表示出来たり、	最適化された経路検索を自動で行うことが可能になる。
		  \item {\bf 天気予報}・・・「天気予報 API(livedoor 天気互換)」というサービスを利用すれば、指定した場所の天気を表示することができる。これは気象庁からの天気データを取得して最高気温や最低気温、降水確率なども表示できる。ただ、調べていくと、気象庁 HP の公に公開されていない API から取得しているため、レスポンスに予期しないデータが入っていた場合や API のデータ構造が変更された場合などに、エラーで取得できなくなる可能性があるとのこと。
 	\end{itemize}
\end{quote}
\par このようにAPIは自分で開発することなく、色々なサービスを活用することができる。
\par では、実際にAPIを利用して何かアプリケーションを作成してみようと思う。ただ、初心者にとって、アプリケーション開発はどこから手をつけていいのかがわからない。そんな時に少し面白そうなものを発見したのでここで紹介する。

\section{GASとは}
GASとはGoogle Apps Scriptの略で、Googleが提供する各種サービスの自動化/連携を行うためのローコード開発ツール\footnote{可能な限りソースコードを書かずにアプリケーションを開発する手法}のこと。GASを使うと、Gmailやカレンダー、Googleスプレッドシート、Googleドライブなど、Googleが提供する様々なサービス上で処理を自動化したり、複数のサービスを連携させることができる。Gmailアカウントを保有していれば誰でもGASを利用することができる。
\par GASでプログラムコードを記述際に使用するスクリプト言語は、JavaScriptをベースに作られており、GASの開発環境はWebブラウザベースで動作するため、Webブラウザ上で所定のURLを開くだけですぐに開発作業に着手できる。

\section{GASで業務効率化}
実際にAPIを活用してみる。手始めに{\bf フォーム入力からメールを自動送信する}機能を実装してみる。この機能を実現することで、イベントなどで申し込み完了に関する自動返信メールを送ったり、アンケート回答に対しての自動メールを送ることができる。
\subsection{ワークフローの説明}
今回の機能は、「ユーザーがGoogleFormで申し込みをする」→「申し込み情報をGASで取得」→「申し込み情報をユーザーにメールで送信する」とする。
\subsection{組み込みオブジェクト}
それらの機能を実装するにあたり、組み込みオブジェクトについて説明する。
組み込みオブジェクトは、プログラミング言語の標準ライブラリに組み込まれているオブジェクトのこと。これらのオブジェクトは、プログラムの開発者が新しく定義する必要なく、すぐに使用できる便利な機能を提供する。以下では、一般的なプログラミング言語で使用される組み込みオブジェクトについて説明する。

\subsubsection{数値オブジェクト}
数値オブジェクトは、数学的な操作を実行するための組み込みオブジェクトで、整数、浮動小数点数などが含まる。例えば、Pythonでは以下のように使用することができる。

\begin{verbatim}
x = 5  # 整数
y = 3.14  # 浮動小数点数

# 加算
result = x + y  # resultは8.14
\end{verbatim}

\subsubsection{文字列オブジェクト}
文字列オブジェクトは、文字列の操作やテキスト処理に使用される。文字列の結合、分割、検索、置換などを行うことができる。JavaScriptでの例を示す。

\begin{verbatim}
let str1 = "Hello";
let str2 = "World";

// 文字列の結合
let result = str1 + " " + str2;  // resultは "Hello World"
\end{verbatim}

\subsubsection{配列オブジェクト}
配列オブジェクトは、複数の値を格納するためのデータ構造で、順序がある。要素の追加、削除、検索、ソートなどが可能。JavaScriptでの例を示す。

\begin{verbatim}
let numbers = [1, 2, 3, 4, 5];

// 配列要素へのアクセス
let firstNumber = numbers[0];  // firstNumberは1

// 配列の要素数
let length = numbers.length;  // lengthは5
\end{verbatim}

\subsubsection{辞書オブジェクト}
辞書オブジェクトは、キーと値のペアを関連付けてデータを格納する。キーを使って値を取得し、追加、削除、変更することができる。Pythonでの例を示す。

\begin{verbatim}
person = {"name": "John", "age": 30, "city": "New York"}

# キーを指定して値を取得
name = person["name"]  # nameは"John"

# 新しいキーと値を追加
person["country"] = "USA"
\end{verbatim}

\subsection{その他の組み込みオブジェクト}
上記以外にも、プログラミング言語にはさまざまな組み込みオブジェクトが存在し、日付オブジェクト、正規表現オブジェクト、エラーオブジェクトなどがある。組み込みオブジェクトは、プログラミング言語の基本的な機能を提供し、プログラムの開発を効率化する。各プログラミング言語に固有の組み込みオブジェクトが存在し、それらを活用することで多くのタスクを簡単に実行することができる。
\par 今回の自動メール送信機能では、GASから「sendEmail」オブジェクトを使用することで簡単に機能を実装することが可能となる。

\subsection{2つのスクリプト}
GASの開発方法には、スタンドアロン型とコンテナバインド型の2つがある。以下で、それぞれのスクリプトタイプについて説明する。

\subsubsection{コンテナバインドスクリプト(Container-bound script)}
コンテナバインドスクリプトは、特定のGoogle Appsコンテナに紐づけられており、主な特徴は次の通り。

\begin{itemize}
    \item {\bf 紐づけられたコンテナに依存}・・・ コンテナバインドスクリプトは、特定のGoogle Appsコンテナ(スプレッドシート、ドキュメント、フォーム)に紐づけられてる。つまり、スクリプトはその特定のコンテナ内でのみ実行でき、そのコンテナのデータにアクセスできる。
    
    \item {\bf コンテナのイベントと連携}・・・コンテナバインドスクリプトは、特定のコンテナ内で発生するイベント(例: スプレッドシートのセルが変更されたとき)に応答して実行できる。これにより、カスタム関数、トリガー、カスタムメニューなどを提供するのに便利となる。
    
    \item {\bf コンテナのデータにアクセス}・・・コンテナバインドスクリプトは、紐づけられたコンテナ内のデータにアクセスできる。たとえば、スプレッドシート内のセル値を読み取ったり、変更したりできる。
    
    \item {\bf コンテナ内で共有される}・・・ コンテナバインドスクリプトは、コンテナと共有され、特定のコンテナのメンバーとしてアクセス権を与える。
\end{itemize}

\subsubsection{スタンドアロンスクリプト(Standalone script)}
スタンドアロンスクリプトは、特定のコンテナに依存せず、独立したスクリプトとして存在する。主な特徴は次の通り。

\begin{itemize}
    \item {\bf 独立したスクリプト}・・・ スタンドアロンスクリプトは、特定のコンテナに紐づかず、独立したスクリプトとして存在する。特定のGoogle Appsコンテナに依存しないため、複数のコンテナで同じスクリプトを使用できる。
    
    \item {\bf Webアプリケーションとして公開}・・・ スタンドアロンスクリプトは、Webアプリケーションとして公開でき、外部ユーザーに対してアクセスを許可できる。Webフォームを作成したり、外部データベースと連携するのに適している。
    
    \item {\bf 独自のトリガー}・・・ スタンドアロンスクリプトは、独自のトリガー(時間ベース、イベントベース)を設定できる。特定のコンテナに依存せず、自由にトリガーを設定できる。
    
    \item {\bf コンテナのデータにアクセス}・・・スタンドアロンスクリプトは、特定のコンテナ内のデータには直接アクセスできない。しかし、必要に応じてGoogle AppsのAPIを使用してデータにアクセスできる。
\end{itemize}

\par 簡単にいうと、2つともGASにおける開発方法のこと。コンテナバインドスクリプトは、Googleのサービスと連携するので業務効率化がしやすいが、実装の手間がかかる。スタンドアロンスクリプトは、Googleのサービスと連携しないので、連携する手間が省けるが、開発に制限を受けやすい。今回はGoogleFormで得られた回答を元にするので、コンテナバインドスクリプトで開発していく。

\section{GASでの開発フロー}
Google Apps Script(GAS)は、Google Appsと連携してスクリプトを実行するためのプラットフォームで、カスタム機能の追加やタスクの自動化に利用されている。ここでGASでの開発フローについて触れておく。

\subsection{プロジェクトの作成}
\begin{itemize}
    \item Google Apps Scriptエディタを開き、新しいプロジェクトを作成。
    \item プロジェクトに名前を付け、必要に応じてプロジェクトの説明を追加。
    \item プロジェクトはGoogleドライブ内に保存され、複数のスクリプトファイルを含むことができる。
\end{itemize}

\subsection{スクリプトの記述}
\begin{itemize}
    \item GASエディタ内で、スクリプトファイルを作成し、スクリプトを記述する。
    \item スクリプトはJavaScriptベースで記述され、Google AppsのAPIを使用して連携する。
    \item スクリプト内で関数やトリガーを定義し、イベントハンドラや処理ロジックを実装する。
\end{itemize}

\subsection{テストとデバッグ}
\begin{itemize}
    \item スクリプトの動作をテストし、エラーを特定。GASエディタ内にデバッグツールが提供されており、エラーメッセージや変数の値を確認できる。
    \item ログ出力を使用してスクリプトの動作をモニタリングし、必要に応じてログメッセージを記録する。
\end{itemize}

\subsection{トリガーの設定}
\begin{itemize}
    \item スクリプトを定期的に実行するためのトリガーを設定する。トリガーは時間ベース(特定の間隔で実行)またはイベントベース(特定のイベントが発生したときに実行)で作成できる。
    \item トリガーはGASエディタ内で設定し、実行頻度やタイミングをカスタマイズできる。
\end{itemize}

\subsection{認証とアクセス許可}
\begin{itemize}
    \item スクリプトがGoogle Appsのデータにアクセスする場合、適切な認証とアクセス許可を設定する必要があります。これにはOAuth 2.0認証などが含まれる。
    \item 認証情報やアクセス許可はGASエディタ内で設定し、Google Cloud Platform ConsoleでAPIキーを取得するなどの作業が必要となる。
\end{itemize}

\subsection{本番環境へのデプロイ}
\begin{itemize}
    \item スクリプトが準備ができたら、本番環境で実行するためにプロジェクトをデプロイする。
    \item これには、プロジェクトを公開し、トリガーを設定することが含まれ、公開後にスクリプトは本番環境で実行可能となる。
\end{itemize}

\subsection{監視と保守}
\begin{itemize}
    \item スクリプトが本番環境で実行されている間、定期的にログを確認して問題を監視し、必要に応じて保守作業を行う。
    \item 新しい要件や変更が発生した場合、スクリプトを更新して適応させる。
\end{itemize}

\subsection{ドキュメンテーション}
\begin{itemize}
    \item スクリプトの機能や使用法を文書化し、チームメンバーや他の開発者と共有する。
    \item ドキュメンテーションはスクリプトの理解と保守を容易にする。
\end{itemize}

\subsection{セキュリティ対策}
\begin{itemize}
    \item スクリプト内でセキュリティ対策を実施し、機密データの漏洩を防ぐ。適切なアクセス制御を設定し、データの保護に努める。
\end{itemize}

\subsection{ユーザーサポート}
\begin{itemize}
    \item スクリプトを使用するユーザーからの問い合わせやフィードバックに対応し、サポートを提供する。
    \item ユーザーサポートはスクリプトの正常な運用と満足度を確保する重要な要素となる。
\end{itemize}

GASでの開発は、Google Appsとの連携により多くのカスタマイズと自動化を実現でき、プロジェクトの要件に合わせて、これらのステップをカスタマイズおよび拡張して、効果的なスクリプトを開発することができる。



























%%%%%%%%%%%%%%%%%%%%%%%%%%%%%%%%%%%%%%%%%%%%%%%%%%%%
%二章以降
%%%%%%%%%%%%%%%%%%%%%%%%%%%%%%%%%%%%%%%%%%%%%%%%%%%%
\newpage
\section{最適化問題の例}
\subsection{計画問題}
\begin{quote}
 	\begin{itemize}
 		 \item {\bf 生産・輸送計画問題}は、各工場での生産能力の上限と下限、各小売店での売上、各工場・小売店間の輸送コストがわかっているときに、 どこの工場でどれくらい品物を作り、どこの小売店にどれだけ運べば コストがもっとも小さくなるか、という工場の生産量と輸送ルートを最適化する問題。前期のゼミでは、輸送問題こそ取り上げなかったものの、{\bf 工場における生産問題}という線形計画問題の例を出した。
 		 \item {\bf スケジューリング問題}は、工場などで納期遅れが出ないように、残業が少なくなるように、などの目的関数 を最小化するように決める問題のこと。その他、 時間割や人の割り当てを決めたり、航空会社でのクルーの勤務表を作る問題や、シフト計画問題もスケジューリング問題として扱う。
 	\end{itemize}
\end{quote}

\subsection{配置問題}
\begin{quote}
	 \begin{itemize}
 		 \item {\bf 配置問題}は、資源の分布を目的に応じて最適化する問題。
 		 \item {\bf 施設配置問題}は配置問題の一つで、与えられた空間内において施設をどのように配置するのが望ましいのかについて、利用者の利便性や施設の収益性などの観点から最適化する数理モデル。施設配置問題\cite{配置問題}については数理的に高度な話題ではあるが、身近な例を扱えるので問題設定が具体的にイメージしやすい特徴がある。3.4で施設配置問題の具体例を示す。
	 \end{itemize}
\end{quote}


\subsection{最短路問題}

\begin{quote}
 	\begin{itemize}
 		 \item {\bf 巡回セールスマン問題}とは、セールスマンが任意の出発地点から、訪問しなければいけない全ての都市を一回ずつ訪問して帰ってくる時、1番短い順路(すべての都市を含む単純な閉路\footnote{始点と終点が同じ道(同じ点を2回以上通らずに始点に帰ってくるたどり方)のこと。})を見つける問題のこと。順路の長さは定義次第で色々な問題に応用できる。\\観光地の回り方を例にすると、"{\bf 料金}"について定義にすれば「交通費を1番抑えた時の回り方」、"{\bf 時間}"で定義すれば「一つでも多くのスポットを巡る回り方」、"{\bf 距離}"で定義すれば「最短経路」について考えることができる。\\巡回セールスマン問題については後ほど詳しく扱う。
 	\end{itemize}
\end{quote}

\newpage

\section{施設配置問題}
後期はこの{\bf 施設配置問題}について深く取り上げようと思う。その理由として私自身が日本の地理にとても興味があるのと、最適化問題の中でも取り組みやすい分野であり、自分自身が積極的に取り組めそうだからである。その中で、施設配置問題の実例を踏まえて、今後活用できそうな例を考えていこうと思う。
\subsection{}






\newpage

\begin{thebibliography}{9}
	\bibitem{参考書} 岩永二郎, 石原響太, 西村直樹, 田中一樹 『Pythonで始める数理最適化』
	\bibitem{配置問題} 田中健一, 『施設配置の数理モデル』 \url{https://www.jstage.jst.go.jp/article/bjsiam/23/4/23_KJ00008992858/_pdf}
	
\end{thebibliography}






\end{document}